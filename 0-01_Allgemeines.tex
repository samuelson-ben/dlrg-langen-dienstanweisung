\documentclass[ %
10pt, % Schriftgroesse
a4paper, % Papiergroesse
headsepline, % Kopfzeilenlinie
footsepline, % Fusszeilenlinie
%parskip, % Abstand zw. Absaetzen
headings=normal, % etw. kleinere Ueberschriften
parskip=full,
%draft % Testkram
]{scrreprt}
% packages
\usepackage[utf8x]{inputenc}
\usepackage[T1]{fontenc}
\usepackage{ucs}
\usepackage[ngerman]{babel}
\usepackage{url} % Schönere URLs
\usepackage{graphicx} % Grafiken
%\usepackage{microtype} % Schoenere Blocksatzraender
\usepackage[ngerman]{varioref} % Tolle Verweise
\usepackage[german]{fancyref} % Tolle Verweise V2
\usepackage{hyperref}
%\usepackage{fancyhdr} % Kopf- und Fusszeilen
\usepackage[automark]{scrlayer-scrpage} % Kopf- und Fusszeilen
\usepackage[top=2cm, bottom=4cm, lmargin=2cm,rmargin=2cm, footskip=1.5cm]{geometry} % Seitenraender
\usepackage[locale=DE]{siunitx} % Einheiten
\sisetup{load-configurations = binary} % alsoload=binary is removed in siuntix 3
% Entwurf
\usepackage{color} % Paket für verschiedene Farben
\usepackage{eso-pic}
\usepackage{wallpaper}
\usepackage{tikz}
\usepackage{rotating}
\usepackage{blindtext}

\newcommand{\makedate}{15. April 2013}

\usetikzlibrary{matrix,fadings,calc,positioning,
				decorations.pathreplacing,arrows,shapes}

\definecolor{lightgray}{gray}{.75}

\definecolor{dlrgblue}{cmyk}{.98,.47,0,.3}

\author{}%Einsatz}
%\title{Dienstanweisung}
\title{}
\date{}%Stand: \makedate}
\publishers{}%Landesverband Niedersachsen\\Bezirk Cuxhaven-Osterholz\\Ortsgruppe Langen e.V.}
%\publishers{Ortsgruppe Langen e.V.}


%{
%\AddToShipoutPicture{%
%\AtTextCenter{%
%\makebox(0,0)[c]{\resizebox{\textwidth}{!}{%
%\rotatebox{45}{\textsf{\textbf{\color{lightgray}Entwurf}}}}}
%}
%}
%}

\newcommand{\changefont}[3]{
\fontfamily{#1} \fontseries{#2} \fontshape{#3} \selectfont}

\newcommand{\erganzung}{}
\newcommand{\dienstanweisung}{Übersicht}




\pagestyle{scrheadings}
% \ihead{Technik/Einsatz}
% \ohead{\includegraphics[height=\headheight]{Logo-B-Graustufen}}
\ihead{DLRG OG Langen e.V.}
\ohead{Technik/Einsatz}
\ifoot{\dienstanweisung \erganzung}
\cfoot{Stand: \makedate}
\ofoot{\pagemark}

\begin{document}
\changefont{cmss}{m}{n}
%\maketitle
%\tableofcontents
%\addtocontents{toc}{\protect\thispagestyle{scrheadings}} 
% \addpart{Allgemeines zur Dienstanweisung}
% \thispagestyle{empty}
%\include{Allgemeines}
%\include{WRD}
%\include{WRD-Mat}
%\end{document}
\addchap{Allgemeines zur Dienstanweisung}
\renewcommand{\dienstanweisung}{DA\,0-01/2013}
\thispagestyle{scrheadings}
\addsec{Allgemeines}
Die in der Ortsgruppe (OG) Langen erlassenen Dienstanweisungen (DA) sind als Ergänzungen zu den bestehenden Anweisungen und Richtlinien der Deutschen Lebens-Rettungs-Gesellschaft (DLRG) zu verstehen und sollen in der OG Langen geltende Sonderregelungen und nicht eindeutig geregelte Arbeitsabläufe gliederungsintern festlegen. Sie dienen als Leitfaden zur einheitlichen Verrichtung des Einsatzdienstes in der allgemeinen und besonderen Gefahrenabwehr wie zum Beispiel im Bereich des Wasserrettungsdienstes (WRD) oder des Katastrophenschutzes (KatS).

Die Dienstanweisungen sollen den Einsatzkräften bei der Verrichtung ihrer Dienste und innerhalb der verschiedenen Ausbildungsgänge als Ausbildungs- und Unterrichtshilfe zur Verfügung stehen und von ihnen als Maßstab angesehen und genutzt werden.

Es wurde, da sich keine geeignete Umformulierung anbot, das generische Maskulinum in Ableitungen und Zusammensetzungen als maskuline und feminine Personenbezeichnung beibehalten.\\
Unabhängig davon steht die Ausübung der Funktion selbstverständlich Frauen und Männern gleichermaßen offen.

\addsec{Anwendungsbereich}
Alle in der OG Langen erlassenen Dienstanweisungen finden Anwendung im Bereich der Leitung Ausbildung bzw. der Leitung Einsatz der OG Langen. Dies umfasst sämtliche Kräfte, Material und Ausrüstung.

\addsec{Erlassen von Dienstanweisungen}
Berechtigt zum Erlassen von Dienstanweisungen ist der Vorstand der OG Langen sowie das ihm übergeordnete Gremium der Jahreshauptversammlung. Die Verfassung, Pflege und Veröffentlichung der erlassenen Anweisungen obliegt dem Leiter Ausbildung bzw. dem Leiter Einsatz der OG Langen.

\addsec{Gültigkeit}
Eine Dienstanweisung wird mit dem Tag ihrer Beschlussfassung durch den Vorstand gültig. Das Gültigkeitsdatum wird auf den Dienstanweisungen gesondert ausgewiesen. Für die umgehende Verteilung der erlassenen Dienstanweisungen sind der Leiter Ausbildung bzw. der Leiter Einsatz der OG Langen verantwortlich.

Jede in den von der DA erfassten Bereichen in der OG Langen tätige Person muss die DA gelesen, verstanden und sich mit ihnen einverstanden erklärt haben. Erklärt sich eine Person nicht damit einverstanden, ist es ihr nicht gestattet an den durch die DA (siehe Abschnitt Anwendungsbereich) erfassten Bereichen mitzuarbeiten.

\end{document}