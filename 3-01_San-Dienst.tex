\documentclass[ %
10pt, % Schriftgroesse
a4paper, % Papiergroesse
headsepline, % Kopfzeilenlinie
footsepline, % Fusszeilenlinie
%parskip, % Abstand zw. Absaetzen
headings=normal, % etw. kleinere Ueberschriften
parskip=full,
%draft % Testkram
]{scrreprt}
% packages
\usepackage[utf8x]{inputenc}
\usepackage[T1]{fontenc}
\usepackage{ucs}
\usepackage[ngerman]{babel}
\usepackage{url} % Schönere URLs
\usepackage{graphicx} % Grafiken
%\usepackage{microtype} % Schoenere Blocksatzraender
\usepackage[ngerman]{varioref} % Tolle Verweise
\usepackage[german]{fancyref} % Tolle Verweise V2
\usepackage{hyperref}
%\usepackage{fancyhdr} % Kopf- und Fusszeilen
\usepackage[automark]{scrlayer-scrpage} % Kopf- und Fusszeilen
\usepackage[top=2cm, bottom=4cm, lmargin=2cm,rmargin=2cm, footskip=1.5cm]{geometry} % Seitenraender
\usepackage[locale=DE]{siunitx} % Einheiten
\sisetup{load-configurations = binary} % alsoload=binary is removed in siuntix 3
% Entwurf
\usepackage{color} % Paket für verschiedene Farben
\usepackage{eso-pic}
\usepackage{wallpaper}
\usepackage{tikz}
\usepackage{rotating}
\usepackage{blindtext}

\newcommand{\makedate}{15. April 2013}

\usetikzlibrary{matrix,fadings,calc,positioning,
				decorations.pathreplacing,arrows,shapes}

\definecolor{lightgray}{gray}{.75}

\definecolor{dlrgblue}{cmyk}{.98,.47,0,.3}

\author{}%Einsatz}
%\title{Dienstanweisung}
\title{}
\date{}%Stand: \makedate}
\publishers{}%Landesverband Niedersachsen\\Bezirk Cuxhaven-Osterholz\\Ortsgruppe Langen e.V.}
%\publishers{Ortsgruppe Langen e.V.}


%{
%\AddToShipoutPicture{%
%\AtTextCenter{%
%\makebox(0,0)[c]{\resizebox{\textwidth}{!}{%
%\rotatebox{45}{\textsf{\textbf{\color{lightgray}Entwurf}}}}}
%}
%}
%}

\newcommand{\changefont}[3]{
\fontfamily{#1} \fontseries{#2} \fontshape{#3} \selectfont}

\newcommand{\erganzung}{}
\newcommand{\dienstanweisung}{Übersicht}




\pagestyle{scrheadings}
% \ihead{Technik/Einsatz}
% \ohead{\includegraphics[height=\headheight]{Logo-B-Graustufen}}
\ihead{DLRG OG Langen e.V.}
\ohead{Technik/Einsatz}
\ifoot{\dienstanweisung \erganzung}
\cfoot{Stand: \makedate}
\ofoot{\pagemark}

\begin{document}
\changefont{cmss}{m}{n}
%\maketitle
%\tableofcontents
%\addtocontents{toc}{\protect\thispagestyle{scrheadings}} 
% \addpart{Allgemeines zur Dienstanweisung}
% \thispagestyle{empty}
%\include{Allgemeines}
%\include{WRD}
%\include{WRD-Mat}
%\end{document}
\renewcommand{\dienstanweisung}{DA\,3-01/2013}
\addchap{Sanitätsdienst}
\thispagestyle{scrheadings}
\addsec{Allgemeines}
Im Sanitätsdienst werden hohe Anforderungen an personelle und materielle Ressourcen gestellt. Diese sollen mit dieser Anweisung genau definiert werden.
Grundsätzlich sollte ein sanitätsdienstlicher Einsatz nicht mit einer Besetzung unter vier Personen durchgeführt werden. Eine dieser Personen sollte eine Ausbildung in Führung in der DLRG genossen haben, um die Einsatzkräfte und Einsätze als Einsatzleiter sicher führen zu können. Das Personal wird vom Leiter Einsatz in Rücksprache mit dem Einsatzleiter eingeteilt.

Jede im Sanitätsdienst eingesetzte Person sollte bedenken, dass sie die DLRG nach außen repräsentiert und ihr Verhalten das Ansehen der DLRG bestimmt.


\addsec{Personal}
Im Sanitätsdienst sollten nur Sanitätshelfer (gemäß gültiger PO EH/SAN, dort 3.3.1, nicht älter als drei Jahre) und Sanitäter (gemäß gültiger PO EH/SAN, dort 3.3.2, nicht älter als drei Jahre) eingesetzt werden, die das 16. Lebensjahr vollendet haben.
Die Qualifikation als Sanitätshelfer oder Sanitäter kann durch ein Sanitätstraining (gemäß gültiger PO EH/SAN, dort 3.4.1, nicht älter als zwei Jahre) verlängert werden.
Bei Einsätzen zur Absicherung von (ggf. lange andauernden) Abendveranstaltungen sollten nur Sanitätshelfer und Sanitäter eingesetzt werden, die das 18. Lebensjahr vollendet haben.

Als Sanitäter im o.g. Sinne werden hier auch Personen mit höherwertiger Rettungsdienstauabildung (Rettungssanitäter, Rettungsassistent) und anderes medizinisch ausgebildetes Personal (z.B. Arzt) verstanden.

\addsec{Bekleidung}
Es ist dem Wetter angemessene Einsatzkleidung gemäß DLRG-Standards zu tragen. Es gelten die Bekleidungsanweisungen der Anweisung für den Wasserrettungsdienst.

Im Sanitätsdienst ist grundsätzlich das Rückenschild \glqq Sanitäter\grqq{}, \glqq Sanitätshelfer\grqq{} oder einer ähnlichen sanitätsdienstlichen Qualifikation zu tragen. Genau wie bei Qualifikationsabzeichen dürfen nur Rückenschilder mit der Qualifikation getragen werden, die die Person besitzt. Sind keine geeigneten Rückenschilder vorhanden, ist es gestattet, das Rückenschild \glqq Wasserrettung\grqq{} zu tragen.

%\addsec{Persönliche Ausrüstung}
%Zusätzlich zur angegebenen Bekleidung sollte jede Einsatzkraft im Sanitätsdienst eine Taschenlampe, ein Rettungsmesser, eine Diagnostikleuchte und Notizblock mit Stift in der persönlichen Ausrüstung bei sich tragen.

\addsec{Material}
Im Sanitätseinsatz kommt eine Sanitätstasche zum Einsatz, ausgestattet nach DIN\,13155 (siehe Anhang \vref{appendix:din-13155}).
Zusätzlich ist diese Sanitätstasche nach DLRG-Standards mit Larynxtuben in drei verschiedenen Größen für Erwachsene, einem flexiblen Halswirbelsäulen-Stützkragen und Sauerstoff-Maske, -Brille, -Schlauch, sowie -Reservoir auzustatten. Weiterhin sollte in der Sanitätstasche ein Formular für ein Notfallprotokoll vorliegen.
Vorzuhalten ist außerdem eine ausreichende Menge medizinischer Sauerstoff in einer Sauerstoffflasche mit Druckminderer in transportabler Form und ein AED--Gerät.

\addsec{Sonstiges}
Weiterhin gelten die nicht explizit auf den Wasserrettungsdienst bezogenen Teile der Anweisung für den Wasserrettungsdienst (DA\,4-01) sinngemäß.

\vspace*{\fill}
\begin{tabular}{lcr}
Langen, der \makedate & & \dotfill \\
 & \hspace{4cm} & \hspace{4cm} Unterschrift des Vorsitzenden
\end{tabular}

\appendix
\chapter{DIN 13155}\label{appendix:din-13155}
\thispagestyle{scrheadings}\renewcommand{\erganzung}{ (Anhang)}
\begin{tabular}{r|l}
        \hline
        Menge & Bezeichnung                                                              \\ \hline
        1     & Absaugpumpe                                                              \\ 
        6     & Absaugkatheter (untersch. Größen)                                        \\ 
        1     & Beatmungsbeutel                                                          \\ 
        3     & Beatmungsmasken (untersch. Größen)                                       \\ 
        3     & Guedeltubus (untersch. Größen)                                           \\ 
        1     & Blutdruckmessgerät                                                       \\ 
        1     & Stethoskop                                                               \\ 
        1     & Diagnostikleuchte                                                        \\ 
        16    & Wundschnellverband-EL \SI{10 x 6}{\centi\metre}					         \\ 
        5     & Fingerverband-EL                                                         \\ 
        5     & Fingerverband-WF                                                         \\ 
        10    & Pflasterstrips-WF \SI{1.9 x 7.2}{\centi\metre}					         \\ 
        1     & Verbandpäckchen klein                                                    \\ 
        2     & Verbandpäckchen mittel                                                   \\ 
        2     & Verbandpäckchen groß                                                     \\ 
        2     & Verbandtuch \SI{40 x 60}{\centi\metre}					                 \\ 
        1     & Verbandtuch \SI{60 x 80}{\centi\metre}					                 \\ 
        12    & Wundkompresse \SI{10 x 10}{\centi\metre}				                 \\ 
        2     & Augenkompresse                                                           \\ 
        1     & Rettungsdecke \SI{160 x 210}{\centi\metre}, silber/gold 				 \\ 
        2     & Fixierbinde \SI{6}{\centi\metre}                                         \\ 
        2     & Fixierbinde \SI{8}{\centi\metre}                                         \\ 
        1     & Netzverband Gr. 3                                                        \\ 
        2     & Dreiecktuch Vliesstoff                                                   \\ 
        1     & Erste Hilfe Schere 190                                                   \\ 
        10    & Vliesstofftücher                                                         \\ 
        2     & Folienbeutel \SI{30 x 40}{\centi\metre}					                 \\ 
        8     & Einmalhandschuhe Vinyl                                                   \\ 
        1     & Händedesinfektion \SI{100}{\milli\litre}                                 \\ 
        1     & Splintschiene Standard                                                   \\ 
        2     & Splintschiene Finger                                                     \\ 
        5     & Anhängekarte für Verletzte                                               \\ 
        1     & Heftpflaster \SI{2.50 x 500}{\centi\metre}	    		                 \\
    \end{tabular}


\end{document}